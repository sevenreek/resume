%%%%%%%%%%%%%%%%%
% This is an sample CV template created using altacv.cls
% (v1.6.5, 3 Nov 2022) written by LianTze Lim (liantze@gmail.com). Compiles with pdfLaTeX, XeLaTeX and LuaLaTeX.
%
%% It may be distributed and/or modified under the
%% conditions of the LaTeX Project Public License, either version 1.3
%% of this license or (at your option) any later version.
%% The latest version of this license is in
%%    http://www.latex-project.org/lppl.txt
%% and version 1.3 or later is part of all distributions of LaTeX
%% version 2003/12/01 or later.
%%%%%%%%%%%%%%%%

%% Use the "normalphoto" option if you want a normal photo instead of cropped to a circle
% \documentclass[10pt,a4paper,normalphoto]{altacv}

\documentclass[10pt,a4paper,ragged2e,withhyper]{altacv}
%% AltaCV uses the fontawesome5 and packages.
%% See http://texdoc.net/pkg/fontawesome5 for full list of symbols.

% Change the page layout if you need to
\geometry{left=1.25cm,right=1.25cm,top=1.5cm,bottom=1.5cm,columnsep=1.2cm}

% The paracol package lets you typeset columns of text in parallel
\usepackage{paracol}

% Change the font if you want to, depending on whether
% you're using pdflatex or xelatex/lualatex
\ifxetexorluatex
  % If using xelatex or lualatex:
  \setmainfont{Roboto Slab}
  \setsansfont{Lato}
  \renewcommand{\familydefault}{\sfdefault}
\else
  % If using pdflatex:
  \usepackage[rm]{roboto}
  \usepackage[defaultsans]{lato}
  % \usepackage{sourcesanspro}
  \renewcommand{\familydefault}{\sfdefault}
\fi

% Change the colours if you want to
\definecolor{SlateGrey}{HTML}{2E2E2E}
\definecolor{LightGrey}{HTML}{666666}
\definecolor{DarkPastelRed}{HTML}{450808}
\definecolor{PastelRed}{HTML}{8F0D0D}
\definecolor{GoldenEarth}{HTML}{E7D192}
\colorlet{name}{black}
\colorlet{tagline}{PastelRed}
\colorlet{heading}{DarkPastelRed}
\colorlet{headingrule}{GoldenEarth}
\colorlet{subheading}{PastelRed}
\colorlet{accent}{PastelRed}
\colorlet{emphasis}{SlateGrey}
\colorlet{body}{LightGrey}

% Change some fonts, if necessary
\renewcommand{\namefont}{\Huge\rmfamily\bfseries}
\renewcommand{\personalinfofont}{\footnotesize}
\renewcommand{\cvsectionfont}{\LARGE\rmfamily\bfseries}
\renewcommand{\cvsubsectionfont}{\large\bfseries}


% Change the bullets for itemize and rating marker
% for \cvskill if you want to
\renewcommand{\itemmarker}{{\small\textbullet}}
\renewcommand{\ratingmarker}{\faCircle}

%% Use (and optionally edit if necessary) this .tex if you
%% want to use an author-year reference style like APA(6)
%% for your publication list
% \input{pubs-authoryear}

%% Use (and optionally edit if necessary) this .tex if you
%% want an originally numerical reference style like IEEE
%% for your publication list
\input{pubs-num}

%% sample.bib contains your publications
\addbibresource{main.bib}

\begin{document}
\name{Wojciech Walewski}
\tagline{Fullstack Developer}
%% You can add multiple photos on the left or right
\photoR{2.8cm}{pic}
% \photoL{2.5cm}{Yacht_High,Suitcase_High}

\personalinfo{%
  % Not all of these are required!
  \email{w.walewski112@gmail.com}
  \phone{+48 698 889 299}
  \location{Łódź, Poland}
  \linkedin{wojciech-walewski}
  \github{sevenreek}
  \orcid{0000-0002-9211-7665}
  %% You can add your own arbitrary detail with
  %% \printinfo{symbol}{detail}[optional hyperlink prefix]
  % \printinfo{\faPaw}{Hey ho!}[https://example.com/]
  %% Or you can declare your own field with
  %% \NewInfoFiled{fieldname}{symbol}[optional hyperlink prefix] and use it:
  % \NewInfoField{gitlab}{\faGitlab}[https://gitlab.com/]
  % \gitlab{your_id}
  %%
  %% For services and platforms like Mastodon where there isn't a
  %% straightforward relation between the user ID/nickname and the hyperlink,
  %% you can use \printinfo directly e.g.
  % \printinfo{\faMastodon}{@username@instace}[https://instance.url/@username]
  %% But if you absolutely want to create new dedicated info fields for
  %% such platforms, then use \NewInfoField* with a star:
  % \NewInfoField*{mastodon}{\faMastodon}
  %% then you can use \mastodon, with TWO arguments where the 2nd argument is
  %% the full hyperlink.
  % \mastodon{@username@instance}{https://instance.url/@username}
}

\makecvheader
%% Depending on your tastes, you may want to make fonts of itemize environments slightly smaller
% \AtBeginEnvironment{itemize}{\small}

%% Set the left/right column width ratio to 6:4.
\columnratio{0.6}

% Start a 2-column paracol. Both the left and right columns will automatically
% break across pages if things get too long.
\begin{paracol}{2}
\cvsection{Work Experience}

\cvevent{Fullstack Developer}{HTD Health}{Jul 2022 -- Dec 2022}{}
\begin{itemize}
  \item{Developed dynamic healthcare apps using Ruby on Rails, Sidekiq, React, GraphQL, Redis, PostgreSQL}
\end{itemize}

\divider

\cvevent{Firmware/Software Engineer}{Department of Microelectronics and Computer Science}{Jan 2021 -- Jul 2022}{}
\begin{itemize}
  \item Built Verilog firmware for an ultra-high speed digitizer board to be used in the International Thermonuclear Experimental Reactor (ITER) for plasma diagnostics.
  \item Developed and maintained a multi-threaded data acquistion and control software GUI in C++ with Qt. Optimized a DMA data transfer routine.
  \item Prototyped digital signal processing algorithms using Python and Matlab.
\end{itemize}

\divider

\cvevent{Assistant Software Engineer}{Fujitsu}{Aug 2020 -- Jan 2021}{}
\begin{itemize}
  \item Maintained a remote testing framework for UEFI-based motherboards.
  \item Performed various scripting tasks with Python and bash.
\end{itemize}

\cvsection{Freelance Projects}

\cvevent{Escape room management system}{Mysterious Room}{2019 -- 2020}{}
\begin{itemize}
\item Designed, implemented and maintained a central control system for the most popular escape room of Lodz.
\item Created and deployed a locally hosted web UI for control using Python, Flask and Bootstrap.
\end{itemize}

\divider

\cvevent{Escape room project}{Porta Aenigma}{2019}{}
\begin{itemize}
\item Designed and implemented various puzzles using Arduino microcontrollers and a simple terminal-based central control system.
\item Maintained a puzzle based on the use of a Microsoft Kinect.
\end{itemize}


\newpage


\cvsection{Publications}

%% Specify your last name(s) and first name(s) as given in the .bib to automatically bold your own name in the publications list. 
%% One caveat: You need to write \bibnamedelima where there's a space in your name for this to work properly; or write \bibnamedelimi if you use initials in the .bib

%% You can specify multiple names, especially if you have changed your name or if you need to highlight multiple authors. 
\mynames{Walewski\bibnamedelima Wojciech,
  Walewski,
  Walewski/W.\bibnamedelimi W.}
%% MAKE SURE THERE IS NO SPACE AFTER THE FINAL NAME IN YOUR \mynames LIST

\nocite{*}

\printbibliography[heading=pubtype,title={\printinfo{\faFile*[regular]}{Journal Articles}},type=article]

\divider

\printbibliography[heading=pubtype,title={\printinfo{\faUsers}{Conference Proceedings}},type=inproceedings]

{\it I hereby give consent for my personal data included in my application to be processed for the purposes of the recruitment process.}
%% Switch to the right column. This will now automatically move to the second
%% page if the content is too long.
\switchcolumn

\cvsection{Technologies}

\cvtag{Django}
\cvtag{Flask}
\cvtag{FastAPI}
\cvtag{numpy}\\
\cvtag{Ruby on Rails}
\cvtag{Sidekiq}
\cvtag{GraphQL}\\
\cvtag{React}
\cvtag{ApolloJS}
\cvtag{Svelte}
\cvtag{Redis}

\divider\smallskip

\cvtag{C++}
\cvtag{Qt5}
\cvtag{Verilog}\\
\cvtag{Digital signal processing}

\cvsection{Languages}

\cvskill{English}{5}
\divider

\cvskill{German}{2} %% Supports X.5 values.

%% Yeah I didn't spend too much time making all the
%% spacing consistent... sorry. Use \smallskip, \medskip,
%% \bigskip, \vspace etc to make adjustments.
\medskip

\cvsection{Education}

\cvevent{M.Sc.\ in Computer Science}{Lodz University of Technology (WEEiA)}{Mar 2021 -- Aug 2022}{}
\begin{itemize}
\item Thesis topic: "Real Time Digital Pulse Processing from Radiation Detectors using Field Progammable Gate Arrays"
\item Awarded first place for the best Master thesis of 2022 at WEEiA by SEP.
\end{itemize}

\divider

\cvevent{B.Sc.\ in Telecommunications and Computer Science}{Lodz University of Technology (IFE)}{Oct 2017 -- Feb 2021}{}

\begin{itemize}
\item Thesis topic: "An IoT Smart Storage System". Built a smart storage solution using Django, MQTT, RPi and ESP8266 microcontrollers.
\item Took part in the European Project Semester at Fachhochschule Kiel. Produced a business plan for Edur Pumpenfabrik revolving around Industry 4.0 technologies.
\end{itemize}
% \divider

\end{paracol}


\end{document}
